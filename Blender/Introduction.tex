\documentclass[twocolumn]{article}
\usepackage{graphicx}
\usepackage{amsmath}
\usepackage{url}
\begin{document}
For a system defined with many possible configurations defined through a model Hamiltonian, calculated from model potentials, or through first principles methods it may be desirable to calculate the density of energy states. With the energy density of states the partition function can be exactly calculated an from it many important properties of interest determined, including free energies and specific heats. One method of solving this problem has been temperature dependent simulations involving  Metropolis algorithm and histogram re weighting techniques\cite{landau_MC_simulations}.  Another algorithm called the  Wang and Landau algorithm\cite{WL_phys_rev_lett} has been developed which is temperature independant.  In this paper a method is proposed that combines the use of random sets along with the importance sampling method of the Wang and Landau algorithm. This algorithm is referred to as the ``B$_{L}$ENDER" (B$_{L}$end Each New Density Each Round) algorithm. The name ``B$_{L}$ENDER" functions as an acronym and adjective which comes in part because how it blends the ideas of a random set and the Wang and Landau method, and also due to the nature of the algorithm iteratively blending histograms to produce a converged density of states. The purpose of developing the ``B$_{L}$ENDER" algorithm was to work towards an algorithm that is both accurate and trivial  to parallelize. The Wang and Landau method does have parallel versions but requires restricting random walkers to specific energy ranges. The ``B$_{L}$ENDER" is trivial to parallelize as it is based on a set of random walkers than each can explore the entire energy range. 
\bibliography{Bib}
\bibliographystyle{unsrt}
\end{document}
