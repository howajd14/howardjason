%\documentclass[aps,pre,reprint,superscriptaddress,showkeys]{revtex4-1}
%\documentclass[ aip, jmp, bmf, sd, rsi, amsmath,amssymb, preprint, reprint, author-year, author-numerical,Conference Proceedings]{revtex4-1}
\documentclass{letter}
\usepackage{amsmath}
\usepackage{color}
\usepackage{graphicx}
\usepackage{dcolumn}
\usepackage{verbatim}
\usepackage{epstopdf}
\usepackage{inputenc}
\usepackage{amssymb}

\newcolumntype{d}{D{.}{.}{-1}}





\begin{document}
If we have two  sets of real positive  numbers $\{X_1, ...., X_N\}$ and $\{Y_1, ... , Y_N\}$ and formulate the following identity, 
\begin{equation}
C = \sum_{i=1}^{N} \frac{X_i}{Y_i}\sum_{i=1}^{N}Y_i \;.
\end{equation}

Are there any mathematical theorems related to the case of all $Y_i \gg X_i$?  I am thinking that it should be $C \approx N\sum_{i=1}^{N}X_i$ but I am not sure how to prove this or if there already is a theorem  related to this.
\end{document}
